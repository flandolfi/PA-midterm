\section{Esercizio 3}

\paragraph{\texttt{\textbf{MidTerm}}} Questa classe (\autoref{lst:main})
contiene il {\tt main()} del progetto. Questo utilizza il metodo {\tt init()}
per generare una coppia di chiavi RSA per tre utenti fittizi, Alice, Bob e
Carol, e ad ogni chiave associa un \emph{alias} (ad esempio, la chiave privata e
la chiave pubblica di Alice sono rispettivamente ``SkA'' e ``PkA''). Questi
vengono utilizzati per ricavare le vere chiavi tramite una {\tt HashMap}, mentre
un'altra {\tt HashMap} viene utilizzata per ottenere l'alias da una chiave. Il
metodo termina generando la block chain iniziale, composta da un solo blocco
(\emph{non valido} -- il \emph{mining} dei blocchi deve essere eseguito sempre
manualmente) contenente una coinbase di importo uguale a 10000 per Alice. Il
metodo {\tt main()} prosegue stampando a schermo la lista degli utenti con le
loro chiavi, utilizzando il metodo {\tt printUsers()}, e lo stato attuale della
block chain, con il metodo {\tt printBlockChain()}\footnote{Per migliorare la
leggibilità dell'output, la block chain verrà stampata a schermo solamente dopo
l'esecuzione di un comando che ne modifica il contenuto (ovvero dopo {\tt
transfer}, {\tt mine} o {\tt remove}).}. Successivamente attende l'inserimento
di un comando da input. Oltre ai comandi che sono nella specifica del progetto,
sono supportati anche i comandi {\tt help}, che, tramite il metodo {\tt
printHelp()}, stampa a schermo i comandi supportati e come è possibile
utilizzarli, {\tt users}, che esegue il metodo {\tt printUsers()}, e {\tt
status}, che esegue il metodo {\tt printBlockChain()}. Gli altri comandi
eseguono funzioni già offerte dalle altre classi ma alcuni di questi (ovvero
{\tt report}, {\tt mine} e {\tt transfer}) devono prima effettuare il parsing
degli argomenti per poterle applicare.

\subsection*{Esempio di esecuzione} Partendo dalla block chain iniziale,
effettuiamo una nuova transazione dall'output 0 del blocco 0 (ovvero dalla
coinbase di Alice) per un importo di 10000 da destinarsi a Bob, eseguiamo una
{\tt mine} sul primo e sul secondo blocco, e controlliamo lo stato della block
chain:
%
\begin{lstlisting}[escapeinside={\#}{\#}, numbers=none, frame=none,
basicstyle=\linespread{1.4}\fontsize{10}{8}\ttfamily]
  $> transfer --in=0,0 --out=10000,PkB --sign=SkA,PkA
  #$<omissis>$#
  $> mine 0
  Found nonce value 8214.
  #$<omissis>$#
  $> mine 1
  Found nonce value 7865.
  #$<omissis>$#
  $> check
  The block chain is valid!
\end{lstlisting}
%
Adesso eseguiamo un'altra transazione, sempre dal primo blocco, verso Carol:
%
\begin{lstlisting}[escapeinside={\#}{\#}, numbers=none, frame=none,
basicstyle=\linespread{1.4}\fontsize{10}{8}\ttfamily]
  $> transfer --in=0,0 --out=5000,PkC --sign=SkA,PkA
  #$<omissis>$#
  $> mine 2
  Found nonce value 1546.
  #$<omissis>$#
  $> check
  The block chain is not valid.
\end{lstlisting}
%
A causa di un double payment la block chain non è più valida. Annulliamo quindi
la transazione rimuovendo l'ultimo blocco:
%
\begin{lstlisting}[escapeinside={\#}{\#}, numbers=none, frame=none,
basicstyle=\linespread{1.4}\fontsize{10}{8}\ttfamily]
  $> remove
  Last block removed.
  #$<omissis>$#
  $> check
  The block chain is valid!
\end{lstlisting}
