\section{Esercizio 2}

\paragraph{\texttt{\textbf{BlockChain}}} Con questa classe (\autoref{lst:chain})
è possibile creare una \emph{block chain}. Il costruttore genera un blocco
iniziale (chiamato anche \emph{genesis block}) contenente una \emph{coinbase} e
l'hash del blocco precedente inizializzato a 0. Il metodo {\tt
isValidBlockChain()} controlla che ciascun blocco:
%
\begin{enumerate*}
  \item abbia un hash valido (ovvero che inizi con tre 0);
  \item contenga l'hash del blocco precedente valido;
  \item non contenga transazioni che utilizzino un importo già esaurito (ovvero,
  che non si verifichi un \emph{double payment}).
\end{enumerate*}
%
Quest'ultimo controllo viene eseguito scorrendo la block chain in maniera
ordinata e, per ciascun blocco, viene salvato un {\tt BitSet} con dimensione
uguale al numero di output contenuti nella transazione del blocco. Se uno di
questi output venisse speso in una transazione successiva, il bit di indice
corrispondente a quell'output verrà impostato ad 1. Se una transazione dovesse
spendere un output con bit già ad 1, il metodo ritorna {\tt false}. Il controllo
del valore delle transazioni non viene effettuato perché il design della classe
{\tt Transaction} non permette la creazione di transazioni con valori non
consistenti. Il metodo {\tt getBalance()}\footnote{Ho modificato il nome di
questo metodo da {\tt printBalance()} a {\tt getBalance()}, il quale non stampa
a schermo ma restituisce un oggetto {\tt String}. L'importo verrà poi stampato a
schermo nel metodo {\tt main()} della classe {\tt MidTerm}.} restituisce
l'importo totale posseduto da un dato utente (identificato da una {\tt
PublicKey}). Il calcolo di questo importo viene effettuato scorrendo a ritroso
la block chain e, per ogni transazione contenente tra gli output la {\tt
PublicKey} dell'utente, l'importo a lui erogato viene sommato al totale. Questo
metodo tiene conto degli output già spesi in maniera analoga al metodo
precedente (utilizzando un {\tt BitSet} per ogni transazione). Il metodo {\tt
mine()} esegue il metodo omonimo su un determinato blocco. Per ogni blocco
successivo, viene aggiornato l'hash del blocco precedente e del blocco stesso.
