\section{Esercizio 1}

\paragraph{\texttt{\textbf{Hash}}} Questa classe (\autoref{lst:hash}) ha come
attributi {\tt hash}, un array di {\tt byte} di 32 elementi, e {\tt SHA256}, un
{\tt MessageDigest} statico, generalmente istanziato con algoritmo SHA-256. Un
oggetto può essere creato con un costruttore senza argomenti, il quale
inizializza a 0 gli elementi di {\tt hash}, o con una lista di {\tt Object}. In
tal caso {\tt hash} viene inizializzato con il digest prodotto da {\tt SHA256}
utilizzando la concatenazione degli hashcode degli argomenti del costruttore.
Sono stati aggiunti i metodi {\tt hashcode()}, per poter utilizzare
correttamente un oggetto {\tt Hash} come chiave in una {\tt HashMap}, e {\tt
getShortHash()}, che restituisce un'abbreviazione leggibile di {\tt hash} in un
oggetto {\tt String}.

\paragraph{\texttt{\textbf{Transaction}}} Con questa classe (\autoref{lst:tx}) è
possibile istanziare transazioni o \emph{coinbase} (ovvero transazioni senza
input) mediante il metodo statico {\tt createCoinbase()}. In entrambi i casi
vengono effettuati controlli di consistenza sugli argomenti dei costruttori per
verificare che:
%
\begin{enumerate*}
  \item le chiavi del firmatario siano valide (utilizzando il metodo
  {\tt checkKeyPair()});
  \item l'importo trasferito sia valido e coperto;
  \item il firmatario usi denaro a lui intestato.
\end{enumerate*}

\paragraph{\texttt{\textbf{Block}}} Con questa classe (\autoref{lst:block}) è
possibile creare blocchi destinati ad una block chain. Il metodo privato {\tt
rehash()} ricalcola l'hash del blocco chiamante. Viene utilizzato ogni volta che
viene apportata una modifica al valore di un suo attributo e, in particolare,
nel metodo {\tt mine()}, il quale incrementa il valore di {\tt nonce} ed esegue
{\tt rehash()} finché l'hash prodotto non inizia con tre 0.
